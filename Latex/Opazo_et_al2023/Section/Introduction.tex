% !TEX root = ../journaltemplate.tex

Iron (Fe) stands as the fourth most abundant element in the Earth’s crust. Despite its abundance, the concentrations of iron at the oceanic surface remain relatively low, ranging from 0.1 to 2 nM (nanomolar) (Nakabayashi et al., 2002; Tani et al., 2003). These concentrations have a remarkable impact on the delicate balance of the marine ecosystem. Phytoplankton production, a cornerstone of oceanic food webs and a significant driver of global carbon cycling, can be significantly constrained due to Fe’s crucial enzymatic role (Martin, 1990; Jickells et al., 2005; Martínez-García et al., 2014). Notably influencing processes like photosynthesis, respiration, and nitrogen fixation (Falkowski et al., 1998; Morel and Price, 2003; Kustka et al., 2003), Fe becomes a limiting factor that reverberates through the marine food chain. This influence assumes heightened significance in specific regions, such as the Southern Oceans, the Pacific and North Atlantic, and the Eastern Equatorial Pacific, collectively known as high-nutrient low-chlorophyll concentration (HNLC) zones (Coale et al., 1996; Boyd et al., 2000, 2004, 2007). 

Several pathways contribute to delivering this essential micronutrient to the ocean’s surface. Hydrothermal vents, rivers, glaciers, icebergs, continental edges, and upwelling mechanisms all play roles in introducing iron into oceanic ecosystems (Ducklow et al., 2003; Tagliabue et al., 2016). Nevertheless, beyond continental margins, the importance of dust fluxes, sourced from arid and semi-arid desert regions, emerges as a fundamental factor (Tagliabue et al., 2017; Lambert et al., 2021). These dust fluxes, mobilized by wind forces and sometimes traveling substantial distances, settle onto the ocean surface through dry or wet deposition, thereby becoming a vital avenue for Fe input. The propensity for dust generation lies in regions characterized by low vegetation coverage and water deficits (Prospero and Lamb, 2003; Prospero et al., 2002; Jickells et al., 2005; Mahowald et al., 2005; Buseck and Adachi, 2008; Hand et al., 2003).

This relationship between dust and carbon balances was first illuminated by the work of Gran et al. (1931), followed by the research of Martin (1990) in the Southern Oceans. These studies revealed the pronounced influence of dust events on primary productivity. Since then, several works (Kohfeld et al., 2005; Jaccard et al., 2013; Petit et al., 1990; Steffensen, 1997; Lambert et al., 2008; Archer et al., 2000) postulate that the efficiency of the soft tissue biological pump during glacial periods could be attributed to the increased availability of aeolian iron, thus linking iron availability to CO2 levels. This mechanism, operating as a recurrent Earth system feedback, is believed to have exerted periodic influences on the carbon cycle over the past 800,000 years. This could potentially explain up to one-third of the observed natural variability in CO2 concentrations, ranging approximately between 180 ppmv and 280 ppmv during glacial and interglacial periods, respectively (Petit et al., 1990; Siegenthaler et al., 2005; Lüthi et al., 2008).

Approximately 3\% of atmospheric dust consists of Fe (Marcotte et al., 2020), contributing around 14-16 Tg annually in the form of mineral-sourced dust aerosols (Jickells et al., 2005; Gao et al., 2003). Regrettably, only a fraction of this deposited Fe, ranging from 1 to 10\%, is available to support phytoplankton growth (Journet et al., 2008; Jickells and Spokes, 2001; Archer and Johnson, 2000; Bopp et al., 2003). Fe can exist in two oxidation states – Fe(II) and Fe(III), as organic ferrous (Fe2+) or organic ferric (Fe3+), respectively. Among these, only the ferrous form is bioavailable, although it is less prevalent. The solubility of Fe, typically defined as the amount of metal that passes through a 0.2 or 0.4 µm filter, depends on factors encompassing deposited dust mineralogy, acidity, water pH, and other environmental variables (Luo et al., 2005; Sholkovitz et al., 2012; Marcotte et al., 2020). While Fe(III) dominates due to processes like oxidation, reduction, and photochemical interactions, it exhibits limited solubility (Wells et al., 1995; Byrne et al., 2000). However, it can eventually dissolve by diverse mechanisms, such as proton-induced Fe-O bond breakage (Cwiertny et al., 2008), photochemical reduction (Fu et al., 2010), and organic ligand complexation, which is the most prevalent. Organic ligands, arising biologically from water column organic matter, play a significant role in increasing soluble Fe concentrations (Fe2+), facilitating biological uptake, and extending the residence time of bioavailable Fe, thus mitigating precipitation and scavenging within the water column (Baker and Croot, 2010).

From the mid-1990s onward, efforts emerged to simulate CO2 fluctuations on a global scale using numerical models (Johnson et al., 1997). Diverse modeling approaches, spanning from box models to General Circulation Models (GCMs), attempted to elucidate the complex causes underlying CO2 variability between glacial and interglacial periods. Yet, despite considerable advancement, the intricacies of this variability remained incompletely understood. Consequently, biogeochemical processes, intricately interwoven with ocean-atmosphere dynamics, have been integrated to offer additional insights into CO2 dynamics (Flato et al., 2014).

Our objective is to elucidate the impact of iron solubility in glacial-interglacial atmospheric CO2 balances over the past 800,000 years. Employing cGENIE, an intermediate complexity model emphasizing the carbon cycle, we conducted sensitivity simulations for pre-industrial and Last Glacial Maximum (LGM) periods. Leveraging a diverse array of simulations, models, empirical dust deposition data, and heterogeneous Fe solubility fields, our study hones in on regions highly sensitive to iron biogeochemistry, particularly the HNLC zones. Through this integrated approach, we aim to enhance our current understanding of this intricate interplay between iron solubility and atmospheric carbon capture.