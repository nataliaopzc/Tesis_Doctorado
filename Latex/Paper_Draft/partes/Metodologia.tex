 % !TeX root = ../Paper_Draft.tex


 \subsection{Model description}

\citep{van2021iron}

In this work we use a version of GENIE focused on the carbon cycle, cGENIE muffin, free version v0.9.5 (it is hosted in the \href{https://github.com/derpycode/cgenie.muffin}{cGENIE Github repository}). cGENIE is a type of Earth System Model of Intermediate Complexity (EMIC), composed of different modules, each one representing different components of the Earth system. In this work, we use the 3D frictional–geostrophic ocean module, with 16 depth levels and 36 × 36 equal-area horizontal grid. In longitude, it has a compression of 10°, and in latitude it goes from 3.2°, at the Equator, to 19.2°, at high latitudes \citep{edwards2005uncertainties}. We also use the biogeochemistry module \citep{ridgwell2007marine} and, an atmospheric component of the 2D energy–moisture balance model (EMBM) with prescribed climatological wind fields \citep{cao2009role}. The continental distribution and bathymetry are published in \cite{ridgwell2007marine}.\\

 \subsection{Experiment design}

Both phytoplankton and zooplankton require specific concentrations of certain macro and micronutrients that promote their existence and development. For this work, this will be reflected as organic matter production. It will be estimated based on the fixed radio of \cite{redfield1934proportions,redfield1963influence}, where P/C/O$_2$ = 1/106/$-$ 138. However, concerning Fe and phosphorus, the current version of cGENIE considers a collimitation scheme  (Fe/C and P/C). This is due to the high variability in the demand for these nutrients, determined by the biogeochemistry of the environment, the availability of light, water temperature, and the growth stage of the species \toask{insert cite}. In order to make a correct carbon inventory, tracers and preformed nutrients were used following the base configuration of \cite{cao2009role}. In addition to a simplified ligands scheme and their processes similar to that portrayed by \cite{parekh2004modeling,parekh2006physical}. Dissolved Fe input to the system, consumed during biological uptake, will be driven by various dust flux fields deposited in the surface ocean. The residence time of the particulate organic matter that forms and sinks from the surface layer will be directly related to the scavenging rate. Additional supplies of iron associated with sedimentary processes in the ocean, or any other type of source other than wind, will not be considered. \\

Regarding the forcing of the system, we worked with: 1) 6 pairs of dust fluxes fields from the LGM and Holocene period. One of them is empirical data called ``Lambert'' \citep{lambert2015dust}, the rest are CMIP5 model simulations surface dust flux reconstructions called ``Albani'', ``Takemura'', ``Ohgaito'',``MIROC-ESM'' and ``MRI-CGCM3'' \citep{albani2014improved,takemura2009simulation,ohgaito2018effect,sueyoshi2013set,yukimoto2012new} para mayor detalle ver \cite{lambert2021regional}; or more detail see Lambert et al. (2021); and since the solubility of iron is not well known both today and during the past, 2) different iron solubility fields were created for each of these six initial flux field pairs. These latest results from the application of a linear regression model, estimated from the relationship between the Holocene dust flux field of \cite{mahowald2006change} and the calibrated iron solubility field, developed by the marine biogeochemical cycling module of cGENIE, called BIOGEM  \citep{ridgwell2007marine}. Additionally, for each iron solubility reconstruction, we varied its values in only one of the 5 HNLC regions. This alteration is the product of the application of 4 factors that multiply the original values of the iron solubility fields. These factors are 1/2, 2/3, 2, and 3. Thus, for each of the 6 pairs of iron solubility reconstructions, 5 modified iron solubility reconstructions were produced (one for each of the 5 HNLC zones), by 4 scalar factors. Consequently, the model was forced with 240 different global patterns of iron solubility and six dust flow fields to estimate CO2 capture.\\

We initialized by performing 6 different pre-industrial equilibrium experiments (spin-up), in which the atmospheric CO2 concentration have been settled at 278 ppmv compared to the control (free CO2 ), using the cGENIE Earth System model, and runing for 10000 years. (to reach the equilibrium state). We forcing the experiments using Holocene dust and iron solubility fields from differents authors, with the goal of compared the ocean carbon uptake potential, even so, they have the same pCO2 in their spin-up have variable oceanic nutrients inventory and carbon distribution, which will affect the behavior of the biological pump. Starting from the end of each spin-up, we run global and regional sensitivity experiments (see figure \ref{}) \toask{insert reference} for the Holocene and LGM data, and again we allow the model to run 10000 years. finally, we took the mean of the last 500 years of pCO2 estimates from each simulation (240 in total). \\

%%%%%%%%%%%%%%%%%%%%%%%%%%%%%%%%%%%%%%%%%%%%%%%%

%En este trabajo utilizamos una versión de GENIE centrada en el ciclo del carbón, cGENIE muffin, versión libre v0.9.5 (se encuentra alojada en \href{https://github.com/derpycode/cgenie.muffin}{cGENIE Github repository}). cGENIE es un tipo de Earth System Model of Intermediate Complexity (EMIC), compuesto por diferentes módulos, cada uno representando distintas componentes del sistema terrestre, de los cuales, para este trabajo, utilizamos; (1) El módulo 3D frictional–geostrophic ocean, 

%con 16 niveles de profundidad y 36 × 36 equal area horizontal grid. En longitud tiene una compresión de 10° y en latitud va desde los 3.2°, en el Ecuador, hasta los 19.2° \citep{edwards2005uncertainties}. The continental distribution and batimetry se encuentra publicada en \citep{ridgwell2007marine}; (2) el módulo de biogeoquímica \citep{ridgwell2007marine} y; (3) Una componente atmosférica de 2D energy–moisture balance model (EMBM) with prescribed climatological wind fields \citep{cao2009role}.\\

%Tanto el fitoplancton como el zooplancton requieren de concentraciones específicas de ciertos macro y micro nutrientes que promueven su existencia y desarrollo. Para el propósito de este trabajo, ésto se verá reflejado como producción de materia orgánica. Se estimará en base a los radios fijos de \cite{redfield1934proportions,redfield1963influence}, donde P/C/O$_2$ = 1/106/$-$ 138. No obstante, respecto al Fe y al fósforo, la actual versión de cGENIE considera un esquema de colimitación (Fe/C y P/C). Lo anterior, debido a que existe una alta variabilidad en la demanda de estos nutrientes, determinada por la biogeoquímica del ambiente, la disponibilidad de luz, temperatura del agua, y etapa de crecimiento de las especies \citep{} \toask{Falta cita}. Con el propósito de hacer un correcto inventario de carbón, se utilizaron trazadores y nutrientes preformados siguiendo la configuración base de \cite{cao2009role}. Además de un esquema simplificado de ligandos y sus procesos semejante a lo retratado por \cite{parekh2004modeling,parekh2006physical}. La entrada de Fe disuelto al sistema será a partir de la captación biológica impulsada por diversos campos de flujos de polvo depositados en el océano superficial. El tiempo de residencia de la materia orgánica particulada que se forma y hunde tendrá directa relación con la tasa de scavening. No se considerará suministros adicionales de hierro asociadas a procesos sedimentarios del océano, ni otro tipo de fuente que no sea eólica. \\

%Con respecto a los forzantes del sistema, we worked with: 1) 6 pares de campos de flujo de polvo from the LGM and Holocene period. Uno de ellos es empirical data called ``Lambert'' \citep{lambert2015dust}, los restantes son CMIP5 model simulations surface dust flux reconstructions called ``Albani'', ``Takemura'', ``Ohgaito'',``MIROC-ESM'' and ``MRI-CGCM3'' \citep{albani2014improved,takemura2009simulation,ohgaito2018effect,sueyoshi2013set,yukimoto2012new} para mayor detalle ver \cite{lambert2021regional}; and since the solubility of iron is not well known both today and during the past, 2) differents iron solubility fields were created for each of these six initial flux field pairs. Éstos últimos resultados de la aplicación de un modelo de regresión lineal, estimado a partir de la relación entre el  campo de flujo de polvo del Holoceno de \cite{mahowald2006change} y el campo calibrado de solubilidad del hierro, desarrollado por el marine biogeochemical cycling module of cGENIE, called BIOGEM \citep{ridgwell2007marine}.  Adicionalmente, para cada reconstrucción de solubilidad del hierro, variamos sus valores sólo en una de las 5 regiones HNLC. Esta alteración es producto de la aplicación de 4 factores que multiplican los valores originales de los campos de solubilidad del hierro. Estos factores son: 1/2, 2/3, 2 y 3. Por lo tanto, para cada una de los 6 pares de reconstrucciones de solubilidad de hierro, se produjeron 5 reconstrucciones de solubilidad de hierro modificadas (una por cada una de las 5 zonas HNLC), por 4 factores escalares. Consequently, el modelo fue forzado con 240 diferentes patrones globales de solubilidad del hierro y seis campos de flujo de polvo para estimar la captura de CO2.\\



