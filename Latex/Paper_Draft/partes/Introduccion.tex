 % !TeX root = ../Paper_Draft.tex

\begin{comment}
\linenumbers

During the last 800,000 years, transitions between glacial, cold, and interglacial, warm, periods have shown that the concentration of atmospheric CO2 has varied approximately between 180 ppmv and 280 ppmv, respectively \citep{petit1990palaeoclimatological, siegenthaler2005stable,luthi2008high}. This stadial natural variability shows that internal Earth system feedback mechanisms act periodically to influence the carbon cycle.\\

The latest report from the Intergovernmental Panel on Climate Change \citep{IPCCSixthReport}, highlights the fundamental role of the oceans as containers of energy and greenhouse gases. In particular, concerning carbon dioxide content, it stores approximately fifty times more than the atmosphere and almost twenty times more than the terrestrial biosphere \citep{broecker1980modeling,zeebe2012history}. Thus, part of the glacial-interglacial difference in CO2, between 80 to 100 ppm, could be related to processes of ocean-atmosphere exchange, such as ocean circulation \citep{stephens2000influence}, stratification \citep{francois1997contribution} and nutrient concentration \citep{odalen2020variable}. In this work, we discuss a mechanism that has been invoked to explain at least 1/3 of this variability, the soft tissue biological pump \citep{sigman2010polar, hain2014distinct, yamamoto2018long}. The process by which an exchange of CO2 is promoted between the surface ocean and the underlying atmosphere due to a greater photosynthetic fixation catalyzed by the production and sinking of Particulate Organic Carbon (POC), eventually remineralized, which is stored in the deep ocean as carbon dissolved inorganic (DIC).\\

 Starting with \cite{gran1931conditions} work and then with \cite{martin1990glacial} research in the Southern Oceans, the relationship between dust episodes and their important impact on primary productivity was shown. Thus, it is believed that the soft tissue biological pump could have been more efficient during glacial periods \citep{archer2000caused}. Both, due to the greater availability of aeolian iron \citep{kohfeld2005role,jaccard2013two},and because colder surface waters can retain more carbon dioxide due to the lower oxygen concentration, which increases solubilization \citep{jaccard2012large}.\\

Iron (Fe) is the fourth most common element in the earth's crust. However, it has concentrations at the oceanic surface of the order of 0.1 a 2 nM \citep{nakabayashi2002variation,tani2003iron}, low enough to become a limiting element of phytoplankton production \citep{martin1990glacial,jickells2005global, martinez2014iron}, due to an enzymatic role that affects processes such as photosynthesis, respiration, and nitrogen fixation \citep{falkowski1998biogeochemical, morel2003biogeochemical,kustka2003revised}. It is particularly important in certain areas, such as the Southern Oceans, the Pacific and North Atlantic, and the Eastern Equatorial Pacific, known as high-nutrient low-chlorophyll concentration (HNLC) regions \citep{coale1996massive,boyd2000mesoscale,boyd2004decline,boyd2007mesoscale}. However, there are various routes by which this micronutrient can eventually reach the surface ocean, such as hydrothermal vents, rivers, glaciers, icebergs, continental edges, upwelling \citep{ducklow2003role,tagliabue2016well}. However, far from the continental margins, dust fluxes have been shown to play a fundamental role \citep{tagliabue2017integral,lambert2021regional}. These fluxes, originating in arid and semi-arid desert areas \citep{prospero2003african, prospero2002environmental,jickells2005global} product of the wind effort on the earth's surface, can be transported long distances, although This will depend on the size of the particle from its source \citep{mahowald2005atmospheric, buseck2008nanoparticles}, before being deposited in dry, wet or by gravitational settlement \citep{hand2003estimates}. Dust is susceptible to being generated in regions with low vegetation cover and with a water deficit \citep{mahowald2005atmospheric}. It, could explain the connection between dust and CO2 balances during glacial periods and, in particular, during the Last Glacial Maximum, approximately 21,000 years ago before the present, since there was a greater load of atmospheric dust \citep{petit1990palaeoclimatological, steffensen1997size, lambert2008dust, archer2000caused}, becoming from 2 to 20 times higher in certain regions \citep{mahowald1999dust, ohgaito2018effect, lambert2015dust}. This increase in dust deposition could be the consequence of an increase in the intensity of the winds, an extension of the ice fields and with it, a greater glaciogenic source and a reduced hydrological cycle \citep{reader1999mineral,mahowald1999dust,lunt2002dust,mahowald2005atmospheric}.\\

Around 3\% of the dust load is Fe \citep{marcotte2020mineral}, with a concentration of around 14-16 Tg annually from mineral source dust aerosols \citep{jickells2005global,gao2003aeolian}. However, of all the Fe deposited, only 1 to 10\% manages to be available for phytoplankton  \citep{journet2008mineralogy, jickells2001atmospheric,archer2000model, bopp2003dust}. Fe can exist in two oxidation states Fe(II) and Fe(III), organic ferrous (Fe2+ ) or organic ferric (Fe3+), respectively. However, only its ferrous form is bioavailable, although less frequently. The solubility of Fe, typically defined as the amount of metal that passes through a 0.2 or 0.4 µm filter, will depend on both the mineralogy and acidity of the deposited dust and the pH of the water \citep{luo2005estimation,sholkovitz2012fractional,marcotte2020mineral}. The Fe(III) found in a greater proportion as a product of processes such as oxidation, reduction, and photochemical processes but it is the less soluble \cite{wells1995iron,byrne2000iron}. However, it can eventually dissolve through certain processes such as the breaking of Fe-O bonds by protons \citep{cwiertny2008characterization}, photochemical reducing effect \citep{fu2010photoreductive} and, the most frequent, organic complexation through organic ligands, a molecule produced biologically from organic matter in the water column \cite{kuma1996controls,nakabayashi2002variation,tani2003iron}. Now, there are two types of organic complexification in the ocean: a strong (L1), through ligands that bind to Fe, an example would be the presence of siderophores in the water \citep{baker2010atmospheric} and; another weak (L2), produced by the presence of nanoparticles formed by the disintegration of organic matter, also known as colloids, which, due to their smaller size, increase their interaction with the organic chelating agents and, therefore, with the solvent \citep{barbeau2001photochemical,hunter2007iron}. The effect of organic complexing is essential both in increasing the soluble Fe (Fe2+ ) concentration, improving biological absorption and in increasing the residence times of bioavailable Fe, avoiding the precipitation and scavenging in the water column  \citep{baker2010atmospheric}. Regardless of the importance of soluble iron, there is great uncertainty around determining the atmospheric inputs of bioavailable iron to the oceans, on the one hand, due to the variability of dust fluxes, on the other hand, to the unknown quantity of Fe solubilized from the deposition. For this reason, many studies devoted to ocean biogeochemical simulations have assumed a spatially constant Fe solubility  \citep{fung2000iron,parekh2004modeling,parekh2006physical,moore2004upper,keith2006nitrogen}.  However, it is logical to think that given the diverse conditions of the different ocean basins in terms of the number of nutrients, temperatures, acids, among others, the solubility of iron is not spatially constant \citep{mahowald2005atmospheric}. Thus, studying this effect could be crucial to determine the role of this micronutrient in the biological pump, therefore, in atmospheric pCO2.\\

Since the mid-1990s, atmospheric variations in CO2 began to be simulated using numerical models on a global scale \citep{johnson1997controls}. Various types of high and low complexity modeling have tried to achieve this goal, such as; box models, highly efficient but with a tendency to homogenize ocean volumes \citep{weber2010utility};  intermediate complexity coupled models, for instance, the Earth System Model of intermediate complexity (EMIC), with reduced parameterization but which operate at time scales that can exceed 10,000 years and; high-performance models that have a higher computational cost associated, as the General Circulation Models (GCM) (3-D models) \citep{flato2014evaluation}. However, the causes of CO2 variability between glacial and interglacial times remain poorly understood. For this reason, biogeochemistry has been incorporated as an important additional process to the physical processes of ocean-atmosphere dynamics. \\

Our goal is to highlight the current understanding of the effect that iron solubility has had on glacial-interglacial atmospheric CO2 balances. For this purpose, we conducted pre-industrial (PI) and LGM sensitivity simulations using cGENIE, an intermediate complexity model focused on the carbon cycle. For this purpose, we will use a set of different simulations and empirical dust deposition data, and heterogeneous Fe solubility fields, for both periods. Additionally, we focused here on regions highly sensitive to iron biogeochemistry such as the HNLC zones.

\end{comment}

Iron (Fe) stands as the fourth most abundant element in the Earth's crust. Despite its abundance, the concentrations of iron at the oceanic surface remain relatively low, ranging from 0.1 to 2 nM (nanomolar) \citep{nakabayashi2002variation,tani2003iron}. These concentrations have a remarkable impact on the delicate balance of the marine ecosystem. Phytoplankton production, a cornerstone of oceanic food webs and a significant driver of global carbon cycling, can be significantly constrained due to Fe's crucial enzymatic role \citep{martin1990glacial,jickells2005global,martinez2014iron}. Notably influencing processes like photosynthesis, respiration, and nitrogen fixation \citep{falkowski1998biogeochemical,morel2003biogeochemical,kustka2003revised}, Fe becomes a limiting factor that reverberates through the marine food chain. This influence assumes heightened significance in specific regions, such as the Southern Oceans, the Pacific and North Atlantic, and the Eastern Equatorial Pacific, collectively known as high-nutrient low-chlorophyll concentration (HNLC) zones \citep{coale1996massive,boyd2000mesoscale,boyd2004decline,boyd2007mesoscale}.

Several pathways contribute to delivering this essential micronutrient to the ocean's surface. Hydrothermal vents, rivers, glaciers, icebergs, continental edges, and upwelling mechanisms all play roles in introducing iron into oceanic ecosystems \citep{ducklow2003role,tagliabue2016well}. Nevertheless, beyond continental margins, the importance of dust fluxes, sourced from arid and semi-arid desert regions, emerges as a fundamental factor \citep{tagliabue2017integral,lambert2021regional}. These dust fluxes, mobilized by wind forces and sometimes traveling substantial distances, settle onto the ocean surface through dry or wet deposition, thereby becoming a vital avenue for Fe input. The propensity for dust generation lies in regions characterized by low vegetation coverage and water deficits \citep{prospero2003african,prospero2002environmental,jickells2005global,mahowald2005atmospheric,buseck2008nanoparticles,hand2003estimates}.

This relationship between dust and carbon balances was first illuminated by the work of \cite{gran1931conditions}, followed by the research of \cite{martin1990glacial} in the Southern Oceans. These studies revealed the pronounced influence of dust events on primary productivity. Since then, several works \citep{kohfeld2005role,jaccard2013two,petit1990palaeoclimatological,steffensen1997size,lambert2008dust,archer2000caused} postulate that the efficiency of the soft tissue biological pump during glacial periods could be attributed to the increased availability of aeolian iron, thus linking iron availability to CO2 levels. This mechanism, operating as a recurrent Earth system feedback, is believed to have exerted periodic influences on the carbon cycle over the past 800,000 years. This could potentially explain up to one-third of the observed natural variability in CO2 concentrations, ranging approximately between 180 ppmv and 280 ppmv during glacial and interglacial periods, respectively \citep{petit1990palaeoclimatological,siegenthaler2005stable,luthi2008high}.

Approximately 3\% of atmospheric dust consists of Fe \citep{marcotte2020mineral}, contributing around 14-16 Tg annually in the form of mineral-sourced dust aerosols \citep{jickells2005global,gao2003aeolian}. Regrettably, only a fraction of this deposited Fe, ranging from 1 to 10\%, is available to support phytoplankton growth \citep{journet2008mineralogy,jickells2001atmospheric,archer2000model,bopp2003dust}. Fe can exist in two oxidation states – Fe(II) and Fe(III), as organic ferrous (Fe2+) or organic ferric (Fe3+), respectively. Among these, only the ferrous form is bioavailable, although it is less prevalent. The solubility of Fe, typically defined as the amount of metal that passes through a 0.2 or 0.4 µm filter, depends on factors encompassing deposited dust mineralogy, acidity, water pH, and other environmental variables \citep{luo2005estimation,sholkovitz2012fractional,marcotte2020mineral}. While Fe(III) dominates due to processes like oxidation, reduction, and photochemical interactions, it exhibits limited solubility \citep{wells1995iron,byrne2000iron}. However, it can eventually dissolve by diverse mechanisms, such as proton-induced Fe-O bond breakage \citep{cwiertny2008characterization}, photochemical reduction \citep{fu2010photoreductive}, and organic ligand complexation, which is the most prevalent. Organic ligands, arising biologically from water column organic matter, play a significant role in increasing soluble Fe concentrations (Fe2+), facilitating biological uptake, and extending the residence time of bioavailable Fe, thus mitigating precipitation and scavenging within the water column \citep{baker2010atmospheric}.

From the mid-1990s onward, efforts emerged to simulate CO2 fluctuations on a global scale using numerical models \citep{johnson1997controls}. Diverse modeling approaches, spanning from box models to intricate General Circulation Models (GCMs), attempted to elucidate the complex causes underlying CO2 variability between glacial and interglacial periods. Yet, despite considerable advancement, the intricacies of this variability remained incompletely understood. Consequently, biogeochemical processes, intricately interwoven with ocean-atmosphere dynamics, have been integrated to offer additional insights into CO2 dynamics \citep{flato2014evaluation}.

Our objective is to elucidate the impact of iron solubility on glacial-interglacial atmospheric CO2 balances over the past 800,000 years. Employing cGENIE, an intermediate complexity model emphasizing the carbon cycle, we conducted sensitivity simulations for pre-industrial and Last Glacial Maximum (LGM) periods. Leveraging a diverse array of simulations, models, empirical dust deposition data, and heterogeneous Fe solubility fields, our study hones in on regions highly sensitive to iron biogeochemistry, particularly the HNLC zones. Through this integrated approach, we aim to enhance our current understanding of this intricate interplay between iron solubility and atmospheric carbon capture.


%%%%%%%%%%%%%%%%%%%%%%%%%%%%%%%%%%%%%%%%%%%%%%%%%%%%%%%%%%%%%%%%%%% Español %%%%%%%%%%%%%%%%%%%%%%%%%%%%%%%%
%Durante los últimos 800000 años, tranciciones entre periodos glaciares, fríos, e interglaciares, cálidos, han mostrado que la concentración de CO2 atmosférico ha variado aproximadamente entre 180 ppmv y 280 ppmv, respectivamente (Petit et al., 1999; Siegenthaler et al., 2005; Luthi et al., 2008). Esta estable variabilidad natural muestra que existen mecanismos de retroalimentación internos del sistema terrestre que actuán ciclicamente para influenciar en el ciclo del carbono. 

%El último reporte del Panel Intergubernamental del Cambio Climático (IPCC, 2021), remarca el rol fundamental que tienen los océanos como contenedores de energía y gases de efecto invernadero. En particular, con respecto al contenido de dióxido de carbono, almacena aproximadamente cincuenta veces más que la atmósfera y casi veinte veces más que la biósfera terrestre (Broecker et al., 1980, zeebe2012history). Así, parte de la diferencia glacial-interglacial de CO2 entre 80 a 100 ppm podría estar relacionada con procesos propios del intercambio océano-atmósfera, tales como, la circulación oceánica (Ferrari et al., 2014) \test{Falta agregar paper relacionados con la circulación y CO2}, la estratificación (Francois et al., 1997) y la concentración de nutrientes (). En este trabajo discutimos un mecanismo que se ha invocado para explicar al menos 1/3 de esta variabilidad, la bomba biológica de tejidos blandos (Sigman et al., 2010, Hain et al., 2014, Yamamoto et al., 2018); proceso por el cual se impulsa un intercambio de CO2 entre el oceano superficial y la atmósfera subyacente debido a una mayor fijación fotosintética catalizada por la producción y hundimiento de Carbono Orgánico Particulado (POC), eventualmente remineralizado, que se almacena en el océano profundo como carbono inorgánico disuelto (DIC). A partir de los trabajos de Grant, y luego con la investigación de J. Martin en los Océanos del Sur se mostró la relación entre los episodios de polvo y su importante impacto en la productividad primaria. Así, se cree que la bomba biológica de tejidos blandos podría haber sido más eficienciente durante periodos glaciares (Archer et al., 2000), por un lado, debido a que aguas superficiales más frías pueden retener más dióxido de carbono debido a la menor concentracíón de oxígeno que aumenta la solubilización (Jaccard and Galbraith, 2012), y por otro lado, por la mayor disponibilidad de hierro eólico (Kohfeld et al., 2005, Jaccard et al., 2013).

%El Hierro (Fe) es el cuarto elemento más común en la corteza terrestre, sin embargo, tiene concentraciones en la superficie oceánica del orden de ... suficientemente bajas para transformarse en un elemento limitante de la producción de fitoplancton (Martin, 1990, Jickells et al., 2005, Martínez García et al., 2014), debido a rol enzimático que afecta procesos como la fotosíntesis, respiración y fijación del nitrógeno (Falkowski et al., 1998; Morel and Price, 2003  3). Lo anterior es particularmente importante en ciertas zonas, tales como, los Océanos del Sur, el Pacífico y Atlántico Norte, y el Pacífico Ecuatorial Este, conocidas como high-nutrient low-chlorophyll concentration (HNLC) regions (Coale et al., 1996,Boyd et al., 2000; Boyd et al., 2004, Boyd et al.,2007  3). Sin embargo, existen diversos rutas por la cual este micronutriente puede llegar eventualmente al océano supérficial, tales como, fuentes hidrotermales, ríos, glaciares, témpanos de hielo, bordes continentales, surgencia /cite{}. No obstante, lejos de los márgenes continentales los flujos de polvo han demostrado jugar un papel fundamental /cite{Tagliabue et al., 2017} /test{poner estudios de experimentos realizados}. Estos flujos, originados en zonas desérticas áridas y semiáridas (Prospero and Lamb, 2003,Prospero et al., 2002 4,Jickells et al., 2005  3) producto del esfuerzo del viento sobre la superficie terrestre, pueden ser transportados largas distancias, aunque esto dependerá del tamaño de la partícula (paper4), desde su fuente antes de ser depositados en forma seca, húmeda o por asentamiento gravitacional (paper4). Son especialmente sensible a generarse en regiones de baja cubierta vegetacional y con déficit de agua (paper4). Lo anterior, podría explicar la conexión entre el polvo y los balances de CO2 durante los periodos glaciares y, en particular durante el Último Máximo Glacial, hace aproximadamente 21000 años antes del presente, dado que hubo una mayor carga de polvo atmosférico (Petit et al., 1990,Steffensen, 1997,fabrice, Archer et al., 2000  4), llegando a ser desde 2 hasta 20 veces mayor en ciertas regiones (Mahowald y col., 1999, Ohgaito et al., 2018, paper depositación). Este incremento en la depositación de polvo podría ser consecuencia de un incremento en la intensidad de los vientos, una extención de los campos de hielo y con ello, una mayor fuente glaciogénica y, un reducido ciclo hidrológico (Reader et al., 1999; Mahowald y col., 1999; Lunt y Valdes, 2002;paper4). \\

%Al rededor del 3\% de la carga de polvo es Fe \citep{marcotte2020mineral}, con un concentración en torno a los 14-16 Tg anualmente de fuente mineral dust aerosols \citep{jickells2005global,gao2003aeolian}. Sin embargo, de la totalidad de Fe depositado, tan sólo entre el 1 al 10\% logra estar disponible para el fitoplancton \citep{journet2008mineralogy, jickells2001atmospheric,archer2000model, bopp2003dust}. El Fe puede existir en dos estados de oxidación Fe(II) y Fe(III), ferroso orgánico (Fe2+ ) o férrico orgánico (Fe3+), respectivamente. No obstante, sólo su forma ferrosa es biodisponible aunque menos frecuente. La solubilidad del Fe, definida tipicamente como la cantidad de metal que pasa por un filtro de 0.2 o 0.4 µm, dependerá tanto de la mineralogía y acidez del polvo depositado como del pH de agua \citep{luo2005estimation,sholkovitz2012fractional,marcotte2020mineral}. El Fe(III) que se encuentra en mayor proporción producto de procesos como la oxidación, reducción y procesos fotoquímicos es el menos soluble \cite{wells1995iron,byrne2000iron}. No obstante, éste puede eventualmente disolverse a través de ciertos procesos, tales como; (1) la ruptura de enlaces Fe-O por protones \citep{cwiertny2008characterization}; (2) efecto reductor fotoquímico \citep{fu2010photoreductive} y, la más frecuente; (3) complejización orgánica por medio de ligandos orgánicos, molécula producida biológicamente a partir de la materia orgánica en la columna de agua \cite{kuma1996controls,nakabayashi2002variation,tani2003iron}. Ahora bien, existen dos tipos de complejización orgánica en el océano: una llamada fuerte (L1), a trevés de ligandos que se unen al Fe, ejemplo sería la presencia de siderófos en el agua \citep{baker2010atmospheric} y; otra débil (L2), producida por la presencia de nanopartículas formadas por disgregación de materia orgánica, también conocidas como coloides, que debido a su menor tamaño aumentan su interacción con los quelante orgánicos y, por tanto, con el solvente \citep{barbeau2001photochemical,hunter2007iron}. El efecto de la complejización orgánica es esencial tanto en incrementar la concentración Fe soluble (Fe2+ ), mejorando la absorción biológica, como en incrementar los tiempos de residencia del Fe biodisponible, evitando la precipitación y barrido del Fe en la columna de agua \citep{baker2010atmospheric}. A pesar de la importancia del hierro soluble, existe una gran incertidumbre en torno a determinar las entradas atmosféricas de hierro biodisponible a los océanos, por un lado, debido a la variavilidad de los flujos de polvo, por otro lado, al desconocimiento de la cantidad de Fe solubilizado proveniente de la depositación. Por esta razón, muchos estudios dedicados a hacer ocean biogeochemical simulations han asumido una solubilidad del Fe espacialmente constante \citep{fung2000iron,parekh2004modeling,parekh2006physical,moore2004upper,keith2006nitrogen}. Sin embargo, es lógico pensar que dadas las diversas condiciones de las distintas cuencas oceánicas en términos de cantidad de nutrientes, temperaturas, acides, entre otros, la solubilidad del hierro no sea espacialmente constante \citep{mahowald2005atmospheric}. Así, estudiar este efecto podría ser crucial para determinar el rol de este micronutriente en la bomba biológica, por tanto, en el pCO2 atmosférico. \\

%Desde mediados de los años 90s, se comenzaron a simular las variaciones atmosféricas de CO2 mediante modelos numéricos a escala global \citep{johnson1997controls}. Diversos tipos de modelados de alta y baja complejidad han intentado lograr este cometido, como por ejemplo; los modelos de caja, altamente eficientes pero con tendencia a homogeneizar los volúmenes oceánicos \citep{weber2010utility}; modelos acoplados de complejidad intermedia como los Earth System Model of intermediate complexity (EMIC), con una reducida parametrización pero que operan a escalas de tiempo que pueden superar los 10000 años y; modelos de alto rendimiento pero que tienen asociado un mayor costo computacional como los General Circulation Models (GCM) (modelos en 3-D) \citep{flato2014evaluation}. Sin embargo, las causas de la variabilidad del CO2 entre tiempos glaciares e interglaciares permanecen sin ser del todo comprendidas. Por esta razón la biogeoquímica se ha incorporado como un importante proceso adicional a los procesos físicos de la dinámica océano-atmósfera. \\

%Ourgoal is to highlight the current understanding of efecto que la solubilidad del hierro ha tenido en los balances de CO2 atmosférico glaciares-interglaciares. Con este propósito, we conducted pre-industrial (PI) and LGM sensitivity simulations using cGENIE, un modelo de complejidad intermedia centrado en el ciclo de carbono. Con este propósito usaremos un set de diferentes simulations and empirical data de depositación de polvo y, heterogeneos campos de solubilidad del Fe, para ambos periodos. Additionally, we focused here en regiones altamente sensibles a la biogeoquímica del hierro como las zonas HNLC. 

%%Mediante este estudio se explorará el rol que han ejercido los flujos de polvo en la variabilidad del CO2, dada su importante contribución de hierro en la superficie del océano. Con este propósito se utili- zará un EMIC con énfasis en el ciclo del carbón llamado carbon centric Grid Enabled Integrated Earth system model (cGEnIE) (??) para el periodo del Último Máximo Glacial (UMG, aproximadamente 21000 años atrás) y del comienzo del Holoceno (∼12000 años a.p). El modelo se forzará con campos de flujos de polvo obtenidos de diferentes fuentes. Con el propósito de hacer más realista las simulaciones se variarán parámetros dentro del modelo ligados a la circulación oceánica y al ciclo del hierro.