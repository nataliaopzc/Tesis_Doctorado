 % !TeX root = ../Paper_Draft.tex

\begin{comment}
\linenumbers

Since glacial and interglacial periods, iron is and has been a key micronutrient in marine biogeochemical cycling. Although it is the fourth most common element in the earth's crust, its low concentrations and its rapid ability to oxidize at the ocean surface make it a nutrient that limits primary productivity, becoming the cornerstone of carbon exchanges between the atmosphere and the ocean. One of the main sources of iron in the open ocean is the aeolian dust fluxes. These occur over long distances, mainly in arid and semi-arid regions, and can be transported to the ocean by episodes of intense winds. However, only a tiny amount will be able to alleviate the heterogeneous demand for phytoplankton. In order to understand the effect of iron on the natural variability of atmospheric CO$_2$, which has ranged between 180-280 ppm during cold and warm periods, respectively. In this study, through an intermediate complexity model focused on the carbon cycle, cGENIE, we conducted, pre-industrial (PI, 1750) and the Last Glacial Maximum (LGM, 26-19 ky BP), sensitivity experiments of the carbon cycle to variations on the iron solubility. We implemented an iron-phosphorus co-limitation and a simple iron-ligand complexation scheme. Additionally, we focus on different ocean basins especially sensitive to iron fertilization as the high-nutrient low-chlorophyll (HNLC) regions. We forced simulations with empirical data, and CMIP5 model simulations surface dust flux reconstructions; and regression models of global iron solubility fields, for both PI and LGM periods. Our results show ... \\

{\bf Keywords:} glacial and interglacial; iron solubility; primary production; dust fluxes 

\end{comment}

% Due to iron dissolution constrols in seawater are very poorly understood this could give us important insights of what role plays on determinate the carbon concentration.  

% El hierro es y ha sido durante periodos glaciales e interglaciares un micronutriente clave en el océano. Si bien es el cuarto elemento más común en la corteza terrestre, su bajas concentraciones sumado a su rápida capacidad de oxidarse en la superficie del océano hacen que sea un nutriente que limita la productividad primaria, convirtiendose en la piedra angular de los intercambios de carbono entre la atmósfera y el océano. Una de las principales fuentes de hierro al océano abierto son los flujos de polvo atmosféricos. Estos se producen a grandes distancias, principalmente en regiones áridas y semiáridas, y pueden ser transportados hasta el océano por episodios de intensos vientos. Sin embargo, sólo una minúscula cantidad podrá aliviar la heterogenea demanda del fitoplancton. Con el propósito de entender el efecto del hierro en la variabilidad natural del CO2 atmosférico, que ha oscilado entre 180-280 ppm durante periodos fríos y cálidos respectivamente. En este estudio a través de un modelo de complejidad intermedia centrado en el ciclo del carbono, cGENIE, conducimos una serie de experimentos, pre-industriales (PI) y del Último Máximo Glacial (LGM), de la sensibilidad del ciclo del carbono a variaciones en la solubilidad del hierro. Para lo anterior, implementamos una colimitación de hierro-fósforo y un esquema de complejización de hierro-ligandos simple. Adicionalmente, nos centramos en diferentes cuencas oceánicas especialmente sensibles a la fertilización por hierro (zonas HNLC). Forzamos las simulaciones con campos globales de solubilidad del hierro, aplicando cuatro factores de escala, y de depositación de polvo, del PI y LGM. Nuestros resultados muestran...
